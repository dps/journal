%%%%%%%%%%%%%%%%%%%%%%%%%%%%%%%%%%%%%%%%%%%%%%%%%%%%%%%%%%%%%%%%%%%%%
%
%	MONTHLY PLANNER
%
%%%%%%%%%%%%%%%%%%%%%%%%%%%%%%%%%%%%%%%%%%%%%%%%%%%%%%%%%%%%%%%%%%%%%

% LAYOUT:       Odd Page:                   Even Page:
%		<Month>                   ||                <year>
%               --------------------------++----------------------
%               *Notes _| Mon | Tue | Wed || Thu | Fri | Sat | Sun
%		       _|     |     |     ||     |     |     |     <= 5 rows (6th will be crammed in 5th)
%               ..........................||......................
%               prior/next
%               month
%               mini cal.

% Settings-----------------------------------------------------------

% percentage (0.0 - 1.0) taken by the '*Notes' column out of the full quarter textwidth
\newcommand{\percentNoteColWidth}{0.95}
% reminder, easier to set explicitly, must add to 1.0
\newcommand{\percentNoteColWhite}{0.05}

% Column width in monthly planner: make room for rules
\newlength{\DayColWidthMP}
\setlength{\DayColWidthMP}{\textwidth / 4}

% Width of '*Notes' column
\newlength{\NoteColWidthMP}
\setlength{\NoteColWidthMP}{\DayColWidthMP * \real{\percentNoteColWidth}}
\newlength{\NoteColSkipMP}
\setlength{\NoteColSkipMP}{\DayColWidthMP * \real{\percentNoteColWhite}}

% Column Types for Days Columns in Monthly Planner
% - notes
\newcolumntype{N}{>{\centering}p{\NoteColWidthMP}@{\hspace\NoteColSkipMP}}
% - days, first/middle (Odd page: Mon, Tue; Even page: Thu, Fri, Sat)
\newcolumntype{M}{>{\centering}p{\DayColWidthMP}@{\extracolsep\fill}}
% - days, last (Odd page: Wed; Even page: Sun) \centering in combination with \\ doesn't work in last column, use \tabularnewline instead
\newcolumntype{O}{>{\centering\arraybackslash}p{\DayColWidthMP}@{}}

% Parameters for the month mini-calendars
% * Column types
\newlength{\WkdayColWidthMinicalMP}
\setlength{\WkdayColWidthMinicalMP}{\NoteColWidthMP / 7}
%   - weeekdays
\newcolumntype{P}{>{\hfill\bfseries\tiny}p{\WkdayColWidthMinicalMP}@{\extracolsep\fill}}
%   - Saturday (\vstrut is for the benefit of \color)
\newcolumntype{Q}{>{\hfill\bfseries\tiny\vstrut{0pt}\color{WeekendDay}}p{\WkdayColWidthMinicalMP}@{\extracolsep\fill}}
%   - Sunday (\vstrut is for the benefit of \color)
\newcolumntype{R}{>{\hfill\bfseries\tiny\vstrut{0pt}\color{WeekendDay}}p{\WkdayColWidthMinicalMP}}
% Tabular
\newcommand{\MonthMiniCalOnMonthlyPlanner}[2]{%
        \begin{tabular*}{\NoteColWidthMP}[t]{@{}*{5}{P}QR@{}}
		\multicolumn{7}{>{\columncolor{HeadMainBg}}c}{\scriptsize\vstrut{1.1em}\bfseries\color{white}#1} \\
                \rowcolor{HeadSubBg}
                \WkdayTblRow{\hspace*{\fill}}\\
                #2
        \end{tabular*}}

% Notes column on odd page in Monthly Planner
% - hand tunned by trial and error, currently not posssible otherwise:
%   the heigh of the multirow is set up at the current evaluation based on current settings (font, etc)
% - so: artificially reserve (5 rows * 8 height) + 1 rows at current row heigh then build the minipage
% - last "8" reserved for building the previous/next months calendars
% Takes 4 params:
% - #1 previous month name
% - #2 previous month tabular layout
% - #3 next month name
% - #4 next month tabular layout
\newcounter{NotesColumnRows}%
\newcommand{\NoteColumn}[4]{%
	\multirow{41}{\NoteColWidthMP}{%
		\begin{minipage}{\NoteColWidthMP}%
			\makebox[0pt][l]{\smash[b]{\color{WriteBgSec}\rule[-24\baselineskip]{1em}{25\baselineskip}}}%
			{\color{WriteBgMain}\raggedright
				\setcounter{NotesColumnRows}{0}%
				\whiledo{\value{NotesColumnRows}<25}{%
					\rule{\NoteColWidthMP}{\arrayrulewidth}
					\stepcounter{NotesColumnRows}
				}%
			}%
		\end{minipage}%
		\break
		% previous month mini calendar
		\begin{minipage}{\NoteColWidthMP}%
			\renewcommand{\arraystretch}{0.8}
			\MonthMiniCalOnMonthlyPlanner{#1}{#2}
		\end{minipage}%
		\break
		% next month mini calendar
		\begin{minipage}{\NoteColWidthMP}%
			\renewcommand{\arraystretch}{0.8}
			\MonthMiniCalOnMonthlyPlanner{#3}{#4}
		\end{minipage}%
		\break \vspace{4.6pt} % push it to the top
	}%
}

% Cell format on Monthly Planner, pure play :)
% - template: #1 - color for day number and left rule, #2 - date number
\newcommand{\CellFormatMPTemplate}[2]{%
	% gray bar at the top, ~ half row width
	\makebox[0pt][l]{\smash[b]{\color{WriteBgSec}\rule[0.4\baselineskip]{\DayColWidthMP}{0.5\baselineskip}}}%
	% thin rule to the left almost full heigh 
	{\color{#1}\rule[-6\baselineskip]{\arrayrulewidth}{6\baselineskip}}%
	% right-aligned date numbers
	\makebox[2.7ex]{\hspace{\fill}{\color{#1}#2}}%
	% the column is centered so force the previous box to the left
	\hspace*{\fill}%
	\break%
	% white last row, makes the left sided line open/incomplete (does not join the one below)
	\vstrut{1em}}
% this one for cramming 2 days in one table cell, for months spanning 6 rows
% - template: #1 - color for day number and left rule, #2 - first date number, #3 second date number
\newcommand{\CellFormatMPTemplateTwo}[3]{%
	% gray bar at the top, ~ half row width
	\makebox[0pt][l]{\smash[b]{\color{WriteBgSec}\rule[0.4\baselineskip]{\DayColWidthMP}{0.5\baselineskip}}}%
	% thin rule to the left almost full heigh 
	{\color{#1}\rule[-2.5\baselineskip]{\arrayrulewidth}{2.5\baselineskip}}%
	% right-aligned date numbers
	\makebox[2.7ex]{\hspace{\fill}{\color{#1}#2}}%
	% the column is centered so force the previous box to the left
	\hspace*{\fill}%
	\break%
	% white last row, makes the left sided line open/incomplete (does not join the one below)
	\vstrut{1em}%
	% repeat
	\makebox[0pt][l]{\smash[b]{\color{WriteBgSec}\rule[0.4\baselineskip]{\DayColWidthMP}{0.5\baselineskip}}}%
	{\color{#1}\rule[-2.5\baselineskip]{\arrayrulewidth}{2.5\baselineskip}}%
	\makebox[2.7ex]{\hspace{\fill}{\color{#1}#3}}%
	\hspace*{\fill}%
	\break%
	\vstrut{1em}}

% for days in current month
% the optional argument is for months spanning 6 rows
\newcommand{\CellFormatMP}[2][]{%
	\ifthenelse{\equal{#1}{}}%
		{\CellFormatMPTemplate{Black}{#2}}%
		{\CellFormatMPTemplateTwo{Black}{#2}{#1}}}

% for days in previous/next month
\newcommand{\CellFormatMPDisabled}[1]{%
	\CellFormatMPTemplate{DayMPDisabled}{#1}}

% WARNING: The \vstrut in the heading for \MPLeftPage and \MPRightPage ensure enough space
%          (and proper alignmeng of the following table) when accented characters are used

% Monthly Planner Left Page
% takes 6 parameters:
% - #1   full name of current month
% - #2-5 are passed as #1-4 to \NoteColumn (see def)
% - #6   tabular layout for current month: \MP<Mon>Left
\newcommand{\MPLeftPage}[6]{%
	{\bfseries\Huge #1\vstrut[-0.5em]{1.4em}}\par
	\nointerlineskip
	\begin{tabular*}{\textwidth}{NMMO}
		\rowcolor{HeadMainBgMP}%
		% \makebox[1em] centers \ding{93} on the gray left vertical bar below, see \NoteColumn
		\bfseries\vstrut{1.1em}{\color{white}\makebox[1em]{\ding{93}}\ \ding{50}\hfill} & \bfseries\vstrut{1.1em}\color{white}\Monday & \bfseries\vstrut{1.1em}\color{white}\Tuesday & \bfseries\vstrut{1.1em}\color{white}\Wednesday \\
		% insert a dummy row: white, (cant use vspacing on previous \tabularnewline because it will turn colored as previous row)
		\\[-0.8em]
		\NoteColumn{#2}{#3}{#4}{#5}
			#6{\CellFormatMP}{\CellFormatMPDisabled}
	\end{tabular*}
	
	\clearpage
}

% Monthly Planner Right Page (similar to Right Page)
% takes 1 parameters:
% - #1    tabular layout for current month: \MP<Mon>Right
\newcommand{\MPRightPage}[1]{%
	\hfill{\bfseries\Huge\MyYear\vstrut[-0.5em]{1.4em}}\par
	\nointerlineskip

	\begin{tabular*}{\textwidth}{MMMO}
		\rowcolor{HeadMainBgMP}
		\bfseries\vstrut{1.1em}\color{white}\Thursday & \bfseries\vstrut{1.1em}\color{white}\Friday & \bfseries\vstrut{1.1em}\color{white}\Saturday & \bfseries\vstrut{1.1em}\color{white}\Sunday \\
		% insert a dummy row: white, (cant use vspacing on previous \tabularnewline because it will turn colored as previous row)
		\\[-0.8em]
		#1{\CellFormatMP}{\CellFormatMPDisabled}
	\end{tabular*}

	\clearpage
}

% Start--------------------------------------------------------------

\MPLeftPage{\January}{\December}{\MonthTblDecPrev}{\February}{\MonthTblFeb}{\MPJanLeft}
\MPRightPage{\MPJanRight}

\MPLeftPage{\February}{\January}{\MonthTblJan}{\March}{\MonthTblMar}{\MPFebLeft}
\MPRightPage{\MPFebRight}

\MPLeftPage{\March}{\February}{\MonthTblFeb}{\April}{\MonthTblApr}{\MPMarLeft}
\MPRightPage{\MPMarRight}

\MPLeftPage{\April}{\March}{\MonthTblMar}{\May}{\MonthTblMay}{\MPAprLeft}
\MPRightPage{\MPAprRight}

\MPLeftPage{\May}{\April}{\MonthTblApr}{\June}{\MonthTblJun}{\MPMayLeft}
\MPRightPage{\MPMayRight}

\MPLeftPage{\June}{\May}{\MonthTblMay}{\July}{\MonthTblJul}{\MPJunLeft}
\MPRightPage{\MPJunRight}

\MPLeftPage{\July}{\June}{\MonthTblJun}{\August}{\MonthTblAug}{\MPJulLeft}
\MPRightPage{\MPJulRight}

\MPLeftPage{\August}{\July}{\MonthTblJul}{\September}{\MonthTblSep}{\MPAugLeft}
\MPRightPage{\MPAugRight}

\MPLeftPage{\September}{\August}{\MonthTblAug}{\October}{\MonthTblOct}{\MPSepLeft}
\MPRightPage{\MPSepRight}

\MPLeftPage{\October}{\September}{\MonthTblSep}{\November}{\MonthTblNov}{\MPOctLeft}
\MPRightPage{\MPOctRight}

\MPLeftPage{\November}{\October}{\MonthTblOct}{\December}{\MonthTblDec}{\MPNovLeft}
\MPRightPage{\MPNovRight}

\MPLeftPage{\December}{\November}{\MonthTblNov}{\January}{\MonthTblJanNext}{\MPDecLeft}
\MPRightPage{\MPDecRight}

